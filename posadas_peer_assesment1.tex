\documentclass[]{article}
\usepackage{lmodern}
\usepackage{amssymb,amsmath}
\usepackage{ifxetex,ifluatex}
\usepackage{fixltx2e} % provides \textsubscript
\ifnum 0\ifxetex 1\fi\ifluatex 1\fi=0 % if pdftex
  \usepackage[T1]{fontenc}
  \usepackage[utf8]{inputenc}
\else % if luatex or xelatex
  \ifxetex
    \usepackage{mathspec}
  \else
    \usepackage{fontspec}
  \fi
  \defaultfontfeatures{Ligatures=TeX,Scale=MatchLowercase}
\fi
% use upquote if available, for straight quotes in verbatim environments
\IfFileExists{upquote.sty}{\usepackage{upquote}}{}
% use microtype if available
\IfFileExists{microtype.sty}{%
\usepackage{microtype}
\UseMicrotypeSet[protrusion]{basicmath} % disable protrusion for tt fonts
}{}
\usepackage[margin=1in]{geometry}
\usepackage{hyperref}
\hypersetup{unicode=true,
            pdftitle={peer assesment 1},
            pdfborder={0 0 0},
            breaklinks=true}
\urlstyle{same}  % don't use monospace font for urls
\usepackage{color}
\usepackage{fancyvrb}
\newcommand{\VerbBar}{|}
\newcommand{\VERB}{\Verb[commandchars=\\\{\}]}
\DefineVerbatimEnvironment{Highlighting}{Verbatim}{commandchars=\\\{\}}
% Add ',fontsize=\small' for more characters per line
\usepackage{framed}
\definecolor{shadecolor}{RGB}{248,248,248}
\newenvironment{Shaded}{\begin{snugshade}}{\end{snugshade}}
\newcommand{\KeywordTok}[1]{\textcolor[rgb]{0.13,0.29,0.53}{\textbf{#1}}}
\newcommand{\DataTypeTok}[1]{\textcolor[rgb]{0.13,0.29,0.53}{#1}}
\newcommand{\DecValTok}[1]{\textcolor[rgb]{0.00,0.00,0.81}{#1}}
\newcommand{\BaseNTok}[1]{\textcolor[rgb]{0.00,0.00,0.81}{#1}}
\newcommand{\FloatTok}[1]{\textcolor[rgb]{0.00,0.00,0.81}{#1}}
\newcommand{\ConstantTok}[1]{\textcolor[rgb]{0.00,0.00,0.00}{#1}}
\newcommand{\CharTok}[1]{\textcolor[rgb]{0.31,0.60,0.02}{#1}}
\newcommand{\SpecialCharTok}[1]{\textcolor[rgb]{0.00,0.00,0.00}{#1}}
\newcommand{\StringTok}[1]{\textcolor[rgb]{0.31,0.60,0.02}{#1}}
\newcommand{\VerbatimStringTok}[1]{\textcolor[rgb]{0.31,0.60,0.02}{#1}}
\newcommand{\SpecialStringTok}[1]{\textcolor[rgb]{0.31,0.60,0.02}{#1}}
\newcommand{\ImportTok}[1]{#1}
\newcommand{\CommentTok}[1]{\textcolor[rgb]{0.56,0.35,0.01}{\textit{#1}}}
\newcommand{\DocumentationTok}[1]{\textcolor[rgb]{0.56,0.35,0.01}{\textbf{\textit{#1}}}}
\newcommand{\AnnotationTok}[1]{\textcolor[rgb]{0.56,0.35,0.01}{\textbf{\textit{#1}}}}
\newcommand{\CommentVarTok}[1]{\textcolor[rgb]{0.56,0.35,0.01}{\textbf{\textit{#1}}}}
\newcommand{\OtherTok}[1]{\textcolor[rgb]{0.56,0.35,0.01}{#1}}
\newcommand{\FunctionTok}[1]{\textcolor[rgb]{0.00,0.00,0.00}{#1}}
\newcommand{\VariableTok}[1]{\textcolor[rgb]{0.00,0.00,0.00}{#1}}
\newcommand{\ControlFlowTok}[1]{\textcolor[rgb]{0.13,0.29,0.53}{\textbf{#1}}}
\newcommand{\OperatorTok}[1]{\textcolor[rgb]{0.81,0.36,0.00}{\textbf{#1}}}
\newcommand{\BuiltInTok}[1]{#1}
\newcommand{\ExtensionTok}[1]{#1}
\newcommand{\PreprocessorTok}[1]{\textcolor[rgb]{0.56,0.35,0.01}{\textit{#1}}}
\newcommand{\AttributeTok}[1]{\textcolor[rgb]{0.77,0.63,0.00}{#1}}
\newcommand{\RegionMarkerTok}[1]{#1}
\newcommand{\InformationTok}[1]{\textcolor[rgb]{0.56,0.35,0.01}{\textbf{\textit{#1}}}}
\newcommand{\WarningTok}[1]{\textcolor[rgb]{0.56,0.35,0.01}{\textbf{\textit{#1}}}}
\newcommand{\AlertTok}[1]{\textcolor[rgb]{0.94,0.16,0.16}{#1}}
\newcommand{\ErrorTok}[1]{\textcolor[rgb]{0.64,0.00,0.00}{\textbf{#1}}}
\newcommand{\NormalTok}[1]{#1}
\usepackage{graphicx,grffile}
\makeatletter
\def\maxwidth{\ifdim\Gin@nat@width>\linewidth\linewidth\else\Gin@nat@width\fi}
\def\maxheight{\ifdim\Gin@nat@height>\textheight\textheight\else\Gin@nat@height\fi}
\makeatother
% Scale images if necessary, so that they will not overflow the page
% margins by default, and it is still possible to overwrite the defaults
% using explicit options in \includegraphics[width, height, ...]{}
\setkeys{Gin}{width=\maxwidth,height=\maxheight,keepaspectratio}
\IfFileExists{parskip.sty}{%
\usepackage{parskip}
}{% else
\setlength{\parindent}{0pt}
\setlength{\parskip}{6pt plus 2pt minus 1pt}
}
\setlength{\emergencystretch}{3em}  % prevent overfull lines
\providecommand{\tightlist}{%
  \setlength{\itemsep}{0pt}\setlength{\parskip}{0pt}}
\setcounter{secnumdepth}{0}
% Redefines (sub)paragraphs to behave more like sections
\ifx\paragraph\undefined\else
\let\oldparagraph\paragraph
\renewcommand{\paragraph}[1]{\oldparagraph{#1}\mbox{}}
\fi
\ifx\subparagraph\undefined\else
\let\oldsubparagraph\subparagraph
\renewcommand{\subparagraph}[1]{\oldsubparagraph{#1}\mbox{}}
\fi

%%% Use protect on footnotes to avoid problems with footnotes in titles
\let\rmarkdownfootnote\footnote%
\def\footnote{\protect\rmarkdownfootnote}

%%% Change title format to be more compact
\usepackage{titling}

% Create subtitle command for use in maketitle
\newcommand{\subtitle}[1]{
  \posttitle{
    \begin{center}\large#1\end{center}
    }
}

\setlength{\droptitle}{-2em}

  \title{peer assesment 1}
    \pretitle{\vspace{\droptitle}\centering\huge}
  \posttitle{\par}
    \author{}
    \preauthor{}\postauthor{}
    \date{}
    \predate{}\postdate{}
  

\begin{document}
\maketitle

\begin{enumerate}
\def\labelenumi{\arabic{enumi}.}
\tightlist
\item
  Code for reading in the dataset and/or processing the data
\end{enumerate}

\begin{Shaded}
\begin{Highlighting}[]
\CommentTok{#load data}
\NormalTok{data <-}\StringTok{ }\KeywordTok{read.csv}\NormalTok{(}\StringTok{'activity.csv'}\NormalTok{)}
\CommentTok{#complete cases}
\NormalTok{activity <-}\StringTok{ }\NormalTok{data[}\KeywordTok{complete.cases}\NormalTok{(data), ]}
\KeywordTok{head}\NormalTok{(activity)}
\end{Highlighting}
\end{Shaded}

\begin{verbatim}
##     steps       date interval
## 289     0 2012-10-02        0
## 290     0 2012-10-02        5
## 291     0 2012-10-02       10
## 292     0 2012-10-02       15
## 293     0 2012-10-02       20
## 294     0 2012-10-02       25
\end{verbatim}

2.Histogram of the total number of steps taken each day

\begin{Shaded}
\begin{Highlighting}[]
\CommentTok{#calculate steps by day}
\NormalTok{stepsbyday <-}\StringTok{ }\KeywordTok{aggregate}\NormalTok{(steps }\OperatorTok{~}\StringTok{ }\NormalTok{date, activity, sum)}
\NormalTok{stepsbyday}
\end{Highlighting}
\end{Shaded}

\begin{verbatim}
##          date steps
## 1  2012-10-02   126
## 2  2012-10-03 11352
## 3  2012-10-04 12116
## 4  2012-10-05 13294
## 5  2012-10-06 15420
## 6  2012-10-07 11015
## 7  2012-10-09 12811
## 8  2012-10-10  9900
## 9  2012-10-11 10304
## 10 2012-10-12 17382
## 11 2012-10-13 12426
## 12 2012-10-14 15098
## 13 2012-10-15 10139
## 14 2012-10-16 15084
## 15 2012-10-17 13452
## 16 2012-10-18 10056
## 17 2012-10-19 11829
## 18 2012-10-20 10395
## 19 2012-10-21  8821
## 20 2012-10-22 13460
## 21 2012-10-23  8918
## 22 2012-10-24  8355
## 23 2012-10-25  2492
## 24 2012-10-26  6778
## 25 2012-10-27 10119
## 26 2012-10-28 11458
## 27 2012-10-29  5018
## 28 2012-10-30  9819
## 29 2012-10-31 15414
## 30 2012-11-02 10600
## 31 2012-11-03 10571
## 32 2012-11-05 10439
## 33 2012-11-06  8334
## 34 2012-11-07 12883
## 35 2012-11-08  3219
## 36 2012-11-11 12608
## 37 2012-11-12 10765
## 38 2012-11-13  7336
## 39 2012-11-15    41
## 40 2012-11-16  5441
## 41 2012-11-17 14339
## 42 2012-11-18 15110
## 43 2012-11-19  8841
## 44 2012-11-20  4472
## 45 2012-11-21 12787
## 46 2012-11-22 20427
## 47 2012-11-23 21194
## 48 2012-11-24 14478
## 49 2012-11-25 11834
## 50 2012-11-26 11162
## 51 2012-11-27 13646
## 52 2012-11-28 10183
## 53 2012-11-29  7047
\end{verbatim}

\begin{Shaded}
\begin{Highlighting}[]
\CommentTok{# Histogram of steps per day}
\KeywordTok{hist}\NormalTok{(stepsbyday}\OperatorTok{$}\NormalTok{steps, }\DataTypeTok{main =} \StringTok{"Total number of steps per day"}\NormalTok{, }\DataTypeTok{xlab =} \StringTok{"Steps per day"}\NormalTok{)}
\end{Highlighting}
\end{Shaded}

\includegraphics{posadas_peer_assesment1_files/figure-latex/histogram1-1.pdf}
3.Mean and median number of steps taken each day

\begin{Shaded}
\begin{Highlighting}[]
\CommentTok{#calculate mean and median steps by day}
\KeywordTok{mean}\NormalTok{(stepsbyday}\OperatorTok{$}\NormalTok{steps)}
\end{Highlighting}
\end{Shaded}

\begin{verbatim}
## [1] 10766.19
\end{verbatim}

\begin{Shaded}
\begin{Highlighting}[]
\KeywordTok{median}\NormalTok{(stepsbyday}\OperatorTok{$}\NormalTok{steps)}
\end{Highlighting}
\end{Shaded}

\begin{verbatim}
## [1] 10765
\end{verbatim}

\begin{enumerate}
\def\labelenumi{\arabic{enumi}.}
\setcounter{enumi}{3}
\tightlist
\item
  Time series plot of the average number of steps taken
\end{enumerate}

\begin{Shaded}
\begin{Highlighting}[]
\CommentTok{#calculate mean step by interval}
\NormalTok{meanstepsinterval <-}\StringTok{ }\KeywordTok{aggregate}\NormalTok{(steps }\OperatorTok{~}\StringTok{ }\NormalTok{interval, activity, mean)}
\end{Highlighting}
\end{Shaded}

\begin{Shaded}
\begin{Highlighting}[]
\CommentTok{#Time series plot if the average number of steps taken}
\KeywordTok{plot}\NormalTok{(meanstepsinterval}\OperatorTok{$}\NormalTok{interval, meanstepsinterval}\OperatorTok{$}\NormalTok{steps, }\DataTypeTok{type=}\StringTok{'l'}\NormalTok{, }\DataTypeTok{main=}\StringTok{"Time series plot of the average number of steps taken"}\NormalTok{, }\DataTypeTok{xlab=}\StringTok{"Interval"}\NormalTok{, }\DataTypeTok{ylab=}\StringTok{"Mean number of steps"}\NormalTok{)}
\end{Highlighting}
\end{Shaded}

\includegraphics{posadas_peer_assesment1_files/figure-latex/time series 1-1.pdf}

5.The 5-minute interval that, on average, contains the maximum number of
steps

\begin{Shaded}
\begin{Highlighting}[]
\NormalTok{maximunsteps<-}\KeywordTok{which.max}\NormalTok{(meanstepsinterval}\OperatorTok{$}\NormalTok{steps)}
\NormalTok{maximuninterval<-meanstepsinterval[maximunsteps, ]}
\NormalTok{maximunsteps}
\end{Highlighting}
\end{Shaded}

\begin{verbatim}
## [1] 104
\end{verbatim}

\begin{Shaded}
\begin{Highlighting}[]
\NormalTok{maximuninterval}
\end{Highlighting}
\end{Shaded}

\begin{verbatim}
##     interval    steps
## 104      835 206.1698
\end{verbatim}

6.Code to describe and show a strategy for imputing missing data

\begin{Shaded}
\begin{Highlighting}[]
\CommentTok{#1. Calculate and report the total number of missing values in the dataset}
\NormalTok{Missing <-}\StringTok{ }\KeywordTok{length}\NormalTok{(}\KeywordTok{which}\NormalTok{(}\KeywordTok{is.na}\NormalTok{(activity}\OperatorTok{$}\NormalTok{steps)))}


\CommentTok{#2. Devise a strategy for filling in all of the missing values in the dataset.}

\CommentTok{#3. Create a new dataset that is equal to the original dataset but with the missing data filled in.}
\NormalTok{Imputed <-}\StringTok{ }\NormalTok{activity}
\ControlFlowTok{for}\NormalTok{ (i }\ControlFlowTok{in} \DecValTok{1}\OperatorTok{:}\KeywordTok{nrow}\NormalTok{(Imputed)) \{}
    \ControlFlowTok{if}\NormalTok{(}\KeywordTok{is.na}\NormalTok{(Imputed}\OperatorTok{$}\NormalTok{steps[i])) \{}
\NormalTok{        val <-}\StringTok{ }\NormalTok{meanstepsinterval}\OperatorTok{$}\NormalTok{steps[}\KeywordTok{which}\NormalTok{(meanstepsinterval}\OperatorTok{$}\NormalTok{interval }\OperatorTok{==}\StringTok{ }\NormalTok{Imputed}\OperatorTok{$}\NormalTok{interval[i])]}
\NormalTok{        Imputed}\OperatorTok{$}\NormalTok{steps[i] <-}\StringTok{ }\NormalTok{val }
\NormalTok{    \}}
\NormalTok{\}}

\CommentTok{#calculate steps by day}
\NormalTok{stepsbydayI <-}\StringTok{ }\KeywordTok{aggregate}\NormalTok{(steps }\OperatorTok{~}\StringTok{ }\NormalTok{date, Imputed, sum)}
\end{Highlighting}
\end{Shaded}

7.Histogram of the total number of steps taken each day after missing
values are imputed

\begin{Shaded}
\begin{Highlighting}[]
\CommentTok{# Histogram of steps per day}
\KeywordTok{hist}\NormalTok{(stepsbydayI}\OperatorTok{$}\NormalTok{steps, }\DataTypeTok{main =} \StringTok{"Total number of steps per day (imputed)"}\NormalTok{, }\DataTypeTok{xlab =} \StringTok{"Steps per day"}\NormalTok{)}
\end{Highlighting}
\end{Shaded}

\includegraphics{posadas_peer_assesment1_files/figure-latex/histogram2-1.pdf}

\begin{Shaded}
\begin{Highlighting}[]
\KeywordTok{mean}\NormalTok{(stepsbydayI}\OperatorTok{$}\NormalTok{steps)}
\end{Highlighting}
\end{Shaded}

\begin{verbatim}
## [1] 10766.19
\end{verbatim}

\begin{Shaded}
\begin{Highlighting}[]
\KeywordTok{median}\NormalTok{(stepsbydayI}\OperatorTok{$}\NormalTok{steps)}
\end{Highlighting}
\end{Shaded}

\begin{verbatim}
## [1] 10765
\end{verbatim}

8.Panel plot comparing the average number of steps taken per 5-minute
interval across weekdays and weekends

\begin{Shaded}
\begin{Highlighting}[]
\CommentTok{#Create a new factor variable in the dataset with two levels – “weekday” and “weekend” indicating whether a given date is a weekday or weekend day.}


\NormalTok{Imputed}\OperatorTok{$}\NormalTok{date <-}\StringTok{  }\KeywordTok{ifelse}\NormalTok{(}\KeywordTok{as.POSIXlt}\NormalTok{(Imputed}\OperatorTok{$}\NormalTok{date)}\OperatorTok{$}\NormalTok{wday }\OperatorTok\StringTok{ }\KeywordTok{c}\NormalTok{(}\DecValTok{0}\NormalTok{,}\DecValTok{6}\NormalTok{), }\StringTok{'weekend'}\NormalTok{, }\StringTok{'weekday'}\NormalTok{)}

\NormalTok{meanImputed <-}\StringTok{ }\KeywordTok{aggregate}\NormalTok{(steps }\OperatorTok{~}\StringTok{ }\NormalTok{interval }\OperatorTok{+}\StringTok{ }\NormalTok{date, }\DataTypeTok{data=}\NormalTok{Imputed, mean)}
\end{Highlighting}
\end{Shaded}

\begin{Shaded}
\begin{Highlighting}[]
\CommentTok{#Make a panel plot containing a time series plot (i.e. \textbackslash{}color\{red\}\{\textbackslash{}verb|type = "l"|\}type="l") of the 5-minute interval (x-axis) and the average number of steps taken, averaged across all weekday days or weekend days (y-axis). See the README file in the GitHub repository to see an example of what this plot should look like using simulated data.}


\KeywordTok{library}\NormalTok{(ggplot2)}
\end{Highlighting}
\end{Shaded}

\begin{verbatim}
## Warning: package 'ggplot2' was built under R version 3.5.1
\end{verbatim}

\begin{Shaded}
\begin{Highlighting}[]
\KeywordTok{ggplot}\NormalTok{ (meanImputed, }\KeywordTok{aes}\NormalTok{(interval, steps)) }\OperatorTok{+}\StringTok{ }\KeywordTok{geom_line}\NormalTok{() }\OperatorTok{+}\StringTok{ }\KeywordTok{xlab}\NormalTok{(}\StringTok{"5 minute interval"}\NormalTok{) }\OperatorTok{+}\StringTok{ }\KeywordTok{ylab}\NormalTok{(}\StringTok{"average number of steps all weekday days or weekend days "}\NormalTok{)}\OperatorTok{+}\StringTok{ }\KeywordTok{labs}\NormalTok{(}\DataTypeTok{title=} \StringTok{"Time series plot by weekday days or weekend days"}\NormalTok{) }\OperatorTok{+}\StringTok{ }\KeywordTok{facet_grid}\NormalTok{(date }\OperatorTok{~}\StringTok{ }\NormalTok{.)}
\end{Highlighting}
\end{Shaded}

\includegraphics{posadas_peer_assesment1_files/figure-latex/time series 2-1.pdf}


\end{document}
